%!TEX root = main.tex
\titlehead{\centering Pontificia Universidad Javeriana | Facultad de Ingeniería | Departamento de Electrónica}
\title{\Large Reporte Final - Sistema de Gestión de Energía Basado en Almacenamiento en Baterías}
\author{Gerardo de J. Becerra B., M.Sc., Ph.D.}
\date{Bogotá, \today}
% \publishers{Betreut von Dipl.-Ing.~M.~Penkuhn und Dr.-Ing.~M.~Hofmann}

\maketitle

\begin{abstract}
	\noindent
En el presente documento se presenta una descripción del sistema implementado para realizar la gestión de la energía en un sistema de almacenamiento en baterías con capacidad para entregar y recibir energía de la red eléctrica y que tiene la posibilidad de integrar una fuente solar fotovoltaica. El sistema se ha modelado como un problema de programación matemática el cual busca minimizar el costo de la energía por medio de la generación de las señales de referencia que controlan la carga y descarga del banco de baterías para aprovechar la energía disponible en la red y en el sistema fotovoltaico. Se han considerado diferentes esquemas para la definición de los precios de la energía y se implementa un método de respuesta a la demanda para mejorar la eficiencia energética del sistema. Se muestran los detalles de la implementación y los resultados obtenidos para varios escenarios considerados.
El trabajo presentado se ha desarrollado bajo el contrato PUJ-06074-19 con la Pontificia Universidad Javeriana, para el proyecto Minciencias 1102-718-49890: \textit{"Sistema de gestión de energía basado en almacenamiento para la integración de fuentes fotovoltaicas con regulación de voltaje y frecuencia"}.
\end{abstract}

\setcounter{tocdepth}{2}
\tableofcontents
